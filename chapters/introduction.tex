\begin{savequote}[75mm]
Make voyages. Attempt them. There is nothing else.
\qauthor{Tennessee Williams}
\end{savequote}

\chapter{Introduction}
\label{introduction}

\newthought{The Large Hadron Collider and the ATLAS Detector are some of the most impressive feats of the human scientific endeavor.} They are marvels of innovation, design, engineering, and computing, yes, but also of community, collaboration, and curiosity. Many unique physical and software components and an extraordinary group of individuals from diverse backgrounds are necessary to build, maintain, and run these machines and to make productive use of the data they provide.\\

The ATLAS computing system alone contains a remarkable number of unique tools each designed to accommodate task-specific constraints and challenges. These range from hardware triggers that must process nearly 60 million GB of data per second, to reconstruction algorithms that efficiently find patterns is complex, chaotic environments, to physics analyses that seek to precisely separate a rare signal from abundant backgrounds while limiting uncertainty.\\

This is, then, an incredibly exciting research area for students interested in software and computing development. In many cases, it is possible to incorporate novel ideas and innovative methods in this development while still maintaining the advantages and robustness of previous designs. In particular, it is possible to use industry standard Machine Learning techniques in many ATLAS tools. In fact, these techniques are becoming increasingly prevalent within the ATLAS experiment, and in some cases, in particular the planned High Luminosity LHC, they are necessary to address computing challenges.\\

This work is one example of working to solve some of these challenges. It has involved extensive software development to extend current particle ID functionality (thereby enabling a range of new physics searches), to model complex high-dimensional backgrounds, and to develop and enhance Run-2 physics analyses. More broadly, it represents one individual's efforts to help develop improved computing practices and tools within ATLAS and understand the effects of new technologies on High Energy Particle Physics.\\

This thesis includes introductions to the mathematical formulation of particle physics and machine learning as well an overview of the current ATLAS computing systems and challenges it faces. It then describes studies undertaken at the intersection of these three areas to improve physics measurements and search for new processes. 
