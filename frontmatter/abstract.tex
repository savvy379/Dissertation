% the abstract

The Higgs boson was discovered by the ATLAS and CMS collaborations in 2012 using data from $\sqrt{s}$=8TeV proton-proton collisions at the LHC. Since the initial discovery of the $H\rightarrow 4l$ and $H\rightarrow \gamma\gamma$ decays, multiple other Higgs analyses of production modes and decay channels have reached discovery significance. This thesis describes the ongoing search for the still unobserved vector boson ($V=W^{\pm},Z$) associated Higgs production with the Higgs decaying to a tau lepton pair using 139 fb$^{-1}$ of $\sqrt{s}$=13 TeV proton-proton collision data collected by the ATLAS detector during Run-2. This analysis requires the vector boson to decay leptonically in order to reduce background contributions from hadronic activity in the detector and this thesis focuses primarily on the usage of final state electrons in the analysis.\\
\vspace{.1in}
The primary backgrounds in all analysis categories are misidentified (or `faked') objects; these contributions are estimated using a data-driven technique which relies on Machine Learning (ML) for object identification and reconstruction. ML is used broadly in High Energy Physics analyses and this work is introduced, with a focus on techniques for improving electron identification through image processing.